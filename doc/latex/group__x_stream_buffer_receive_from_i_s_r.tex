\hypertarget{group__x_stream_buffer_receive_from_i_s_r}{}\section{x\+Stream\+Buffer\+Receive\+From\+I\+SR}
\label{group__x_stream_buffer_receive_from_i_s_r}\index{xStreamBufferReceiveFromISR@{xStreamBufferReceiveFromISR}}
\mbox{\hyperlink{stream__buffer_8h}{stream\+\_\+buffer.\+h}}


\begin{DoxyPre}
size\_t xStreamBufferReceiveFromISR( StreamBufferHandle\_t xStreamBuffer,
                                    void *pvRxData,
                                    size\_t xBufferLengthBytes,
                                    BaseType\_t *pxHigherPriorityTaskWoken );
\end{DoxyPre}


An interrupt safe version of the A\+PI function that receives bytes from a stream buffer.

Use \mbox{\hyperlink{stream__buffer_8h_a55efc144b988598d84a6087d3e20b507}{x\+Stream\+Buffer\+Receive()}} to read bytes from a stream buffer from a task. Use \mbox{\hyperlink{stream__buffer_8h_a6c882a1d9f26c40f93f271bd1b844b3b}{x\+Stream\+Buffer\+Receive\+From\+I\+S\+R()}} to read bytes from a stream buffer from an interrupt service routine (I\+SR).


\begin{DoxyParams}{Аргументы}
{\em x\+Stream\+Buffer} & The handle of the stream buffer from which a stream is being received.\\
\hline
{\em pv\+Rx\+Data} & A pointer to the buffer into which the received bytes are copied.\\
\hline
{\em x\+Buffer\+Length\+Bytes} & The length of the buffer pointed to by the pv\+Rx\+Data parameter. This sets the maximum number of bytes to receive in one call. x\+Stream\+Buffer\+Receive will return as many bytes as possible up to a maximum set by x\+Buffer\+Length\+Bytes.\\
\hline
{\em px\+Higher\+Priority\+Task\+Woken} & It is possible that a stream buffer will have a task blocked on it waiting for space to become available. Calling \mbox{\hyperlink{stream__buffer_8h_a6c882a1d9f26c40f93f271bd1b844b3b}{x\+Stream\+Buffer\+Receive\+From\+I\+S\+R()}} can make space available, and so cause a task that is waiting for space to leave the Blocked state. If calling \mbox{\hyperlink{stream__buffer_8h_a6c882a1d9f26c40f93f271bd1b844b3b}{x\+Stream\+Buffer\+Receive\+From\+I\+S\+R()}} causes a task to leave the Blocked state, and the unblocked task has a priority higher than the currently executing task (the task that was interrupted), then, internally, \mbox{\hyperlink{stream__buffer_8h_a6c882a1d9f26c40f93f271bd1b844b3b}{x\+Stream\+Buffer\+Receive\+From\+I\+S\+R()}} will set $\ast$px\+Higher\+Priority\+Task\+Woken to pd\+T\+R\+UE. If \mbox{\hyperlink{stream__buffer_8h_a6c882a1d9f26c40f93f271bd1b844b3b}{x\+Stream\+Buffer\+Receive\+From\+I\+S\+R()}} sets this value to pd\+T\+R\+UE, then normally a context switch should be performed before the interrupt is exited. That will ensure the interrupt returns directly to the highest priority Ready state task. $\ast$px\+Higher\+Priority\+Task\+Woken should be set to pd\+F\+A\+L\+SE before it is passed into the function. See the code example below for an example.\\
\hline
\end{DoxyParams}
\begin{DoxyReturn}{Возвращает}
The number of bytes read from the stream buffer, if any.
\end{DoxyReturn}
Example use\+: 
\begin{DoxyPre}
// A stream buffer that has already been created.
StreamBuffer\_t xStreamBuffer;\end{DoxyPre}



\begin{DoxyPre}void vAnInterruptServiceRoutine( void )
\{
uint8\_t ucRxData[ 20 ];
size\_t xReceivedBytes;
BaseType\_t xHigherPriorityTaskWoken = pdFALSE;  // Initialised to pdFALSE.\end{DoxyPre}



\begin{DoxyPre}    // Receive the next stream from the stream buffer.
    xReceivedBytes = xStreamBufferReceiveFromISR( xStreamBuffer,
                                                  ( void * ) ucRxData,
                                                  sizeof( ucRxData ),
                                                  \&xHigherPriorityTaskWoken );\end{DoxyPre}



\begin{DoxyPre}    if( xReceivedBytes > 0 )
    \{
        // ucRxData contains xReceivedBytes read from the stream buffer.
        // Process the stream here....
    \}\end{DoxyPre}



\begin{DoxyPre}    // If xHigherPriorityTaskWoken was set to pdTRUE inside
    // \mbox{\hyperlink{stream__buffer_8h_a6c882a1d9f26c40f93f271bd1b844b3b}{xStreamBufferReceiveFromISR()}} then a task that has a priority above the
    // priority of the currently executing task was unblocked and a context
    // switch should be performed to ensure the ISR returns to the unblocked
    // task.  In most FreeRTOS ports this is done by simply passing
    // xHigherPriorityTaskWoken into taskYIELD\_FROM\_ISR(), which will test the
    // variables value, and perform the context switch if necessary.  Check the
    // documentation for the port in use for port specific instructions.
    taskYIELD\_FROM\_ISR( xHigherPriorityTaskWoken );
\}
\end{DoxyPre}
 