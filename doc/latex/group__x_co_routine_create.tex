\hypertarget{group__x_co_routine_create}{}\section{x\+Co\+Routine\+Create}
\label{group__x_co_routine_create}\index{xCoRoutineCreate@{xCoRoutineCreate}}
croutine. h 
\begin{DoxyPre}
BaseType\_t xCoRoutineCreate(
                                crCOROUTINE\_CODE pxCoRoutineCode,
                                UBaseType\_t uxPriority,
                                UBaseType\_t uxIndex
                              );\end{DoxyPre}


Create a new co-\/routine and add it to the list of co-\/routines that are ready to run.


\begin{DoxyParams}{Аргументы}
{\em px\+Co\+Routine\+Code} & Pointer to the co-\/routine function. Co-\/routine functions require special syntax -\/ see the co-\/routine section of the W\+EB documentation for more information.\\
\hline
{\em ux\+Priority} & The priority with respect to other co-\/routines at which the co-\/routine will run.\\
\hline
{\em ux\+Index} & Used to distinguish between different co-\/routines that execute the same function. See the example below and the co-\/routine section of the W\+EB documentation for further information.\\
\hline
\end{DoxyParams}
\begin{DoxyReturn}{Возвращает}
pd\+P\+A\+SS if the co-\/routine was successfully created and added to a ready list, otherwise an error code defined with \mbox{\hyperlink{projdefs_8h}{Proj\+Defs.\+h}}.
\end{DoxyReturn}
Example usage\+: 
\begin{DoxyPre}
// Co-routine to be created.
void vFlashCoRoutine( CoRoutineHandle\_t xHandle, UBaseType\_t uxIndex )
\{
// Variables in co-routines must be declared static if they must maintain value across a blocking call.
// This may not be necessary for const variables.
static const char cLedToFlash[ 2 ] = \{ 5, 6 \};
static const TickType\_t uxFlashRates[ 2 ] = \{ 200, 400 \};\end{DoxyPre}



\begin{DoxyPre}    // Must start every co-routine with a call to \mbox{\hyperlink{croutine_8h_a19a57a201a325e8af1207ed68c4aedde}{crSTART()}};
    \mbox{\hyperlink{croutine_8h_a19a57a201a325e8af1207ed68c4aedde}{crSTART( xHandle )}};\end{DoxyPre}



\begin{DoxyPre}    for( ;; )
    \{
        // This co-routine just delays for a fixed period, then toggles
        // an LED.  Two co-routines are created using this function, so
        // the uxIndex parameter is used to tell the co-routine which
        // LED to flash and how int32\_t to delay.  This assumes xQueue has
        // already been created.
        vParTestToggleLED( cLedToFlash[ uxIndex ] );
        \mbox{\hyperlink{croutine_8h_a05a06feb11028f2d1d440ea335f562ba}{crDELAY( xHandle, uxFlashRates[ uxIndex ] )}};
    \}\end{DoxyPre}



\begin{DoxyPre}    // Must end every co-routine with a call to \mbox{\hyperlink{croutine_8h_ae6038cc976689b50000475ebfc4e2f23}{crEND()}};
    \mbox{\hyperlink{croutine_8h_ae6038cc976689b50000475ebfc4e2f23}{crEND()}};
\}\end{DoxyPre}



\begin{DoxyPre}// Function that creates two co-routines.
void vOtherFunction( void )
\{
uint8\_t ucParameterToPass;
TaskHandle\_t xHandle;\end{DoxyPre}



\begin{DoxyPre}    // Create two co-routines at priority 0.  The first is given index 0
    // so (from the code above) toggles LED 5 every 200 ticks.  The second
    // is given index 1 so toggles LED 6 every 400 ticks.
    for( uxIndex = 0; uxIndex < 2; uxIndex++ )
    \{
        xCoRoutineCreate( vFlashCoRoutine, 0, uxIndex );
    \}
\}
  \end{DoxyPre}
 